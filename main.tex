\documentclass[12pt]{article}
\usepackage[margin=1in]{geometry} 
\usepackage{amsmath,amsthm,amssymb,amsfonts}
\usepackage{comment}
\usepackage{array}
 
\newcommand{\N}{\mathbb{N}}
\newcommand{\Z}{\mathbb{Z}}
\newcommand{\R}{\mathbb{R}}
\newcommand{\Q}{\mathbb{Q}}
\newcommand{\C}{\mathbb{C}}
\newcommand{\D}{\textbf{D}}
\newcommand\dif{\mathop{}\!\mathtt{d}}
\newcommand{\bvec}[1]{\textbf{#1}}
\renewcommand\labelenumi{\theenumi)}
\renewcommand{\theenumi}{\alph{enumi}}
 
\newenvironment{problem}[2][Problem]{\begin{trivlist}
\item[\hskip \labelsep {\bfseries #1}\hskip \labelsep {\bfseries #2.}]}{\end{trivlist}}
\begin{document}

\title{%
  \large Math 4001 Analysis 2 \\
  Homework Set 3
  }
  \author{Khaled Allen}
\maketitle

\begin{problem}{Apos. 12.19}
Let $\textbf{f}:\R\to\R^2$ be defined by the equation $\textbf{f}(t)=(\cos t, \sin t)$. Then $\textbf{f}'(t)(u)=u(-\sin t, \cos t)$ for every real $u$. Then Mean-Value formula 
\[\textbf{f}(y)-\textbf{f}(x)=\textbf{f}'(z)(y-x)\]
cannot hold when $x=0, y=2\pi$, since the left member is zero and the right member is a vector of length $2\pi$. Nevertheless, Theorem 12.9 states that for every vector $\textbf{a}$ in $\R^2$ there is a $z$ in the inverval $(0,2\pi)$ such that 
\[\textbf{a}\cdot\{\textbf{f}(y)-\textbf{f}(x)\}=\textbf{a}\cdot\{\bvec{f}'(z)(y-x)\}.\]

Determine $z$ in terms of $\textbf{a}$ when $x=0$ and $y=2\pi$.
\end{problem}
\begin{proof}
\begin{align*}
\bvec{a}\left(\begin{bmatrix}1\\0\end{bmatrix}-\begin{bmatrix}1\\
0\end{bmatrix}\right)&=\bvec{a}\cdot2\pi\begin{bmatrix}-\sin z\\\cos z
\end{bmatrix}\\
    0&=2\pi\left(a_2\cos z -a_1\sin z\right)\\
a_1\sin z&=a_2 \cos z\\
\frac{\sin z}{\cos z}&=\frac{a_2}{a_1}\\
z&=\arctan\left(\frac{a_2}{a_1}\right)
\end{align*}
\end{proof}

\begin{problem}{Apos. 12.21}
State and prove a generalization of the result in Exercise 12.20 for a real-valued function differentiable on an $n$-ball $B(x)$.
\end{problem}
\begin{proof}
Let $g(t)$ be defined such that $g(0)=f(\bvec{x}), g(1)=f(\bvec{y})$. Explicitly:
\[g(t)=f[ty_1+(1-t)x_1,y_2,\cdots,y_n]+f[x_1,ty_2+(1-t)x_2,y_3,\cdots,y_n]+\cdots+f[x_1,x_2,\cdots,ty_n+(1-t)x_n]\]

An attempt to express this more succinctly gives \[ g(t)=\sum_i^n f_i\] where $f_k$ is defined such that it's $k$-th argument is $ty_k+(1-t)x_k$ and all arguments before are $x_i$, and all arguments after are $y_i$.

Then,
\begin{align*}
g(1)&=f[(1)y_1+(1-1)x_1,y_2,\cdots,y_n]+f[x_1,(1)y_2+(1-1)x_2,y_3,\cdots,y_n]+\cdots\\&\hspace{.5cm}+f[x_1,x_2,\cdots,(1)y_n+(1-1)x_n]\\
&=f(y_1,y_2,\cdots,y_n)+f(x_1,y_2,y_3,\cdots,y_n)+\cdots+f(x_1,x_2,\cdots,y_n)\\
g(0)&=f[(0)y_1+(1-0)x_1,y_2,\cdots,y_n]+f[x_1,(0)y_2+(1-0)x_2,y_3,\cdots,y_n]+\cdots\\&\hspace{.5cm}+f[x_1,x_2,\cdots,(0)y_n+(1-0)x_n]\\
&=f(x_1,y_2,\cdots,y_n)+f(x_1,x_2,y_3,\cdots,y_n)+\cdots+f(x_1,x_2,\cdots,x_n)
\end{align*}

and we have
\begin{align*}
g(1)-g(0)&=f(y_1,y_2,\cdots,y_n)+f(x_1,y_2,y_3,\cdots,y_n)+\cdots+f(x_1,x_2,x_3\cdots,y_n)\\
&\hspace{1.3in}-f(x_1,y_2,y_3,\cdots,y_n)-f(x_1,x_2,y_3,\cdots,y_n)-\cdots-f(x_1,x_2,\cdots,x_n)\\
&=f(\bvec{y})-f(\bvec{x})
\end{align*}

We then compute the derivatives, noting that for each $f$, we have $n$ partial derivatives, each with respect to each argument of the function $f$. This means that $D_i f_k=D_if[x_1,x_2,\cdots ty_i+(1-t)x_i](y_i-x_i)$ when $i=k$, and $0$ when $i\not=k$.

To illustrate, consider the process applied to just the first summand of $g$:

\begin{align*}
f'[ty_1+(1-t)x_1,y_2,\cdots,y_n]&=D_1f[ty_1+(1-t)x_1,y_2,\cdots,y_n](y_1-x_1)+D_2f[ty_1+(1-t)x_1,y_2,\cdots,y_n](0)+\cdots\\
&\hspace{.5cm}+D_nf[ty_1+(1-t)x_1,y_2,\cdots,y_n](0)\\
&=D_1f[ty_1+(1-t)x_1,y_2,\cdots,y_n](y_1-x_1)
\end{align*}

If we let $z_k(t)=ty_k+(1-t)x_k$, we can write it more simply, and the second summand becomes $D_2f[x_1,z_2,y_3,\cdots,y_n](y_2-x_2)$.
The third: $D_3f[x_1,x_2,z_3,y_4\cdots,y_n](y_3-x_3)$.
And so on. In general the $k$-th summand of $g'(t)$ is given by
\[D_kf_k\cdot(y_k-x_k)\]

Then $g'(t)$ simplifies to
\[
g'(t)=D_1f[z_1,y_2,\cdots,y_n](y_1-x_1)+D_2f[x_1,z_2,y_3,\cdots,y_n](y_2-x_2)+\cdots+D_nf[x_1,\cdots,z_n](y_n-x_n)
\]

By the Mean Value Theorem, with $z_k=z_k(t_0)=t_0y_k+(1-t_0)x_k$

\begin{align*}
    g(1)-g(0)&=g'(t_0)(1-0)\\
    f(\bvec{y}-\bvec{x})&=D_1f[z_1,y_2,\cdots,y_n](y_1-x_1)+D_2f[x_1,z_2,\cdots,y_n](y_2-x_2)+\cdots+D_nf[x_1,\cdots,z_n](y_n-x_n)
\end{align*}
\end{proof}
\vspace{6in}

\begin{problem}{Apos. 13.2}
Let $\bvec{f}=(f_1,f_2,f_3)$ be the vector-valued function defined (for every point $(x_1, x_2, x_2)$ in $\R^3$ for which $x_1+x_2+x_3\not=-1)$ as follows:
\[f_k(x_1,x_2,x_3)=\frac{x_k}{1+x_1+x_2+x_3} k=1,2,3\]
Show that $J_{\bvec{f}}(\bvec{x})=(1+x_1+x_2+x_3)^{-4}$. Show that $\bvec{f}$ is one-to-one and compute $\bvec{f}^{-1}$ explicitly.
\end{problem}

\begin{proof}
    The Jacobian determinant $J_{\bvec{f}}(x)$ is given by
    \begin{align*}
    J_{\bvec{f}}(x)&=
    \begin{vmatrix}
    \frac{\partial f_1}{\partial x_1} & \frac{\partial f_1}{\partial x_2} & \frac{\partial f_1}{\partial x_3} \\
    \frac{\partial f_2}{\partial x_1} & \frac{\partial f_2}{\partial x_2} & \frac{\partial f_2}{\partial x_3} \\
    \frac{\partial f_3}{\partial x_1} & \frac{\partial f_3}{\partial x_2} & \frac{\partial f_3}{\partial x_3} \\
    \end{vmatrix}\\
    &=\frac{\partial f_1}{\partial x_1}\left(\frac{\partial f_2}{\partial x_2}\frac{\partial f_3}{\partial x_3}-\frac{\partial f_2}{\partial x_3}\frac{\partial f_3}{\partial x_2}\right)
    -\frac{\partial f_1}{\partial x_2}\left(
    \frac{\partial f_2}{\partial x_1}\frac{\partial f_3}{\partial x_3}-\frac{\partial f_3}{\partial x_1}
    \frac{\partial f_2}{\partial x_3}
    \right)+\frac{\partial f_1}{\partial x_3}\left(
    \frac{\partial f_2}{\partial x_1}\frac{\partial f_3}{\partial x_2}-
    \frac{\partial f_2}{\partial x_2}\frac{\partial f_3}{\partial x_1}
    \right)\\
    &=\frac{1+x_2+x_3}{(1+x_1+x_2+x_3)^2}\left(\frac{(1+x_1+x_3)(1+x_1+x_2)}{(1+x_1+x_2+x_3)^4}-\frac{(-x_2)(-x_3)}{(1+x_1+x_2+x_3)^4}\right)\\
    &\hspace{.5cm}-\frac{-x_1}{(1+x_1+x_2+x_3)^2}\left(
    \frac{(-x_2)(1+x_1+x_2)}{(1+x_1+x_2+x_3)^4}-\frac{(-x_3)(-x_2)}{(1+x_1+x_2+x_3)^4}
    \right)\\
    &\hspace{.5cm}+\frac{-x_1}{(1+x_1+x_2+x_3)^2}\left(
    \frac{(-x_2)(-x_3)}{(1+x_1+x_2+x_3)^4}-
    \frac{(1+x_1+x_3)(-x_3)}{(1+x_1+x_2+x_3)^4}
    \right)\\
    &=\frac{1+x_2+x_3}{(1+x_1+x_2+x_3)^2}\left(\frac{1+2x_1+x_2+x_1^2+x_1x_2+x_3+x_1x_3+x_2x_3-x_2x_3}{(1+x_1+x_2+x_3)^4}\right)\\
    &\hspace{.5cm}-\frac{-x_1}{(1+x_1+x_2+x_3)^2}\left(
    \frac{-x_2-x_1x_2-x_2^2-x_2x_3}{(1+x_1+x_2+x_3)^4}
    \right)\\
    &\hspace{.5cm}+\frac{-x_1}{(1+x_1+x_2+x_3)^2}\left(
    \frac{x_2x_3+x_3+x_1x_3+x_3^2}{(1+x_1+x_2+x_3)^4}\right)\\
    &=\frac{1}{(1+x_1+x_2+x_3)^6}\left(1+x_1^2+x_2^2+x_3^2+2x_1+2x_2+2x_3+2x_1x_2+2x_3x_2+2x_1x_3\right)\\
    &=\frac{1}{(1+x_1+x_2+x_3)^6}(1+x_1+x_2+x_3)^2\\
    &=\frac{1}{(1+x_1+x_2+x_3)^4}\\
    &=(1+x_1+x_2+x_3)^{-4}
\end{align*}

Since $J_{\bvec{f}}(x)\not=0$ and continuous for any $x_1,x_2,x_3$ for which the function is defined (specifically $x_1+x_2+x_3\not=-1$), the function is one-to-one by the Inverse Function Theorem.

Finally, $\bvec{f}^{-1}(x_1,x_2,x_3)$ is given by
\[
\bvec{f}^{-1}(x_1,x_2,x_3)=\begin{bmatrix}
x_1(1+x_1+x_2+x_3)\\
x_2(1+x_1+x_2+x_3)\\
x_3(1+x_1+x_2+x_3)
\end{bmatrix}
\]

since \[
\bvec{f}^{-1}(\bvec{f}(x))=\bvec{f}^{-1}\left(\begin{bmatrix}
\frac{x_1}{1+x_1+x_2+x_3} \\
\frac{x_2}{1+x_1+x_2+x_3} \\
\frac{x_2}{1+x_1+x_2+x_3}
\end{bmatrix}\right)=
\begin{bmatrix}
\frac{x_1}{1+x_1+x_2+x_3}(1+x_1+x_2+x_3) \\
\frac{x_2}{1+x_1+x_2+x_3}(1+x_1+x_2+x_3) \\
\frac{x_2}{1+x_1+x_2+x_3}(1+x_1+x_2+x_3)
\end{bmatrix}=
\begin{bmatrix}
x_1 \\x_2\\x_3
\end{bmatrix}
=\bvec{x}
\]
\end{proof}

\begin{problem}{Apos. 13.5}
\begin{enumerate}
    \item State conditions on $f$ and $g$ which will ensure that the equations $x=f(u,v), y=g(u,v)$ can be solved for $u$ and $v$ in a neighborhood of $(x_0,y_0)$. If the solutions are $u=F(x,y), v=G(x,y)$, and if $J=\partial(f,g)/\partial(u,v)$, show that
    \[
    \frac{\partial F}{\partial x}=\frac{1}{J}\frac{\partial g}{\partial v},\frac{\partial F}{\partial y}=-\frac{1}{J}\frac{\partial f}{\partial v},    \frac{\partial G}{\partial x}=-\frac{1}{J}\frac{\partial g}{\partial u},    \frac{\partial G}{\partial y}=\frac{1}{J}\frac{\partial f}{\partial u}
    \]
    \item Compute $J$ and the partial derivatives of $F$ and $G$ at $(x_0,y_0)=(1,1)$ when $f(u,v)=u^2-v^2,g(u,v)=2uv$.
\end{enumerate}
\end{problem}

\begin{proof}
\begin{enumerate}
    \item The conditions on $f$ and $g$ that would ensure invertability are that the partial derivatives of each function be continuous in a neighborhood around $(x_0,y_0)$. Also, the Jacobian determinant should not be $0$, which gives us
    \[
        J_{\bvec{f}}(x)=\frac{\partial f}{\partial u}\frac{\partial g}{\partial v}-\frac{\partial f}{\partial v}\frac{\partial g}{\partial u}\not=0
    \]
    This leads to an interesting result, whereby the ratio of partials cannot be equivalent
    \[
        \frac{
            \frac{\partial f}{\partial u}
        }{
            \frac{\partial f}{\partial v}
        }\not=\frac{
            \frac{\partial g}{\partial u}
        }{
            \frac{\partial g}{\partial v}
        }
    \]
    \vspace{6in}
\end{enumerate}
\end{proof}

\begin{problem}{D1}
Given is the function $F(x,y)=x^2+xy+y^2-3.$ Show that $y=f(x)$ in a neighborhood of the point $(1,1)$. Then determine $\partial f/\partial x$ at this point.
\end{problem}

\begin{proof}
Let $\Omega\subset\R^2$. By the IFT, if $F(x,y)=0$ for some $(x,y)\in\Omega$ and the partial derivatives $\frac{\partial F}{\partial x}$ are continuous in a neighborhood around that point, then there exists some $f(x)$ such that $f(x)=y$ and $F(x,f(x))=0$. So we seek to show that $F(x,y)=0$ at $(1,1)$ and that $J$ is not $0$ at (1,1). 
    \[F(1,1)=(1)^2+1\cdot1+(1)^2-3=3-3=0\]
    \[\frac{\partial F}{\partial y}(1,1) = \left.x+2y\right|_{(1,1)}=1+2=3\not=0\]
Thus, by the Implicit Function Theorem, there exists a function $f(x)$ such that $y=f(x)$ in a neighborhood around $(1,1)$. For a larger domain, we can note that, as long as both $x$ and $y$ are not both $0$, the derivative is nonzero. There are some additional restrictions, noting that 
\[y=\sqrt{3-x^2-xy}\]
So that $x^2+xy>3$ to avoid an imaginary result for $y$.

\[\frac{\partial f}{\partial x}(1,1)=\frac{\frac{-\partial F}{\partial x}}{\frac{\partial F}{\partial y}}(1,1)=\left.\frac{2x+y}{x+2y}\right|_{(1,1)}=\frac{2+1}{1+2}=1\]
\end{proof}


    \begin{comment}
    f1x1\frac{1+x_2+x_3}{(1+x_1+x_2+x_3)^2} f1x2\frac{-x_1}{(1+x_1+x_2+x_3)^2} f1x3\frac{-x_1}{(1+x_1+x_2+x_3)^2}
    f2x1\frac{-x_2}{(1+x_1+x_2+x_3)^2} f2x2\frac{1+x_1+x_3}{(1+x_1+x_2+x_3)^2} f2x3\frac{-x_2}{(1+x_1+x_2+x_3)^2}
    f3x1\frac{-x_3}{(1+x_1+x_2+x_3)^2} f3x2\frac{-x_3}{(1+x_1+x_2+x_3)^2} f3x3\frac{1+x_1+x_2}{(1+x_1+x_2+x_3)^2}
    \end{comment}

\begin{comment}
\begin{problem}{x.yz}
Statement of problem goes here
\end{problem}
 
\begin{proof}
Proof goes here. Repeat as needed
\end{proof}
\end{comment}

\end{document}
